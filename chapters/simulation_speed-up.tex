\chapter{Simulation Speed-up}

Understanding the performance gains achieved by utilizing simulated time in a Discrete Event Simulation (DES) model requires a detailed analysis of the cost factors in both real-time and simulated-time scenarios. This chapter explores the key variables influencing system performance and presents formulas to estimate costs and speed-up factors, providing a basis for comparing the efficiency of different simulation techniques.

\section{Key Variables}

To model the simulation's dynamics, the following key variables are defined:

\begin{itemize}
    \item \( p \): The total number of requests initiated by clients throughout the simulation.
    \item \( c \): The number of clients participating in the simulation.
    \item \( n \): The number of possible request types or states a client can have, such as connect, disconnect, or idle.
    \item \( P \): The transition matrix for a Markov chain with dimensions \( n \times n \), representing state transition probabilities that model clients’ behavior.
    \item \( svp \): Stationary distribution vector of the Markov chain, denoted as a \( 1 \times n \) vector, representing the long-term probability distribution of the system’s states.
    \item \( cst \): A vector of dimension \( n \times 1 \) representing the average sleep times of clients in each state.
    \item \( ast \): A vector of dimension \( n \times 1 \) indicating the frequency of send operations performed by clients for each state.
    \item \( art \): A vector of dimension \( n \times 1 \) representing the architecture's average response times for each state.
    \item \( cmot \): A vector of dimension \( n \times 1 \) representing the machine operation times on the client side for each state.
    \item \( amot \): A vector of dimension \( n \times 1 \) representing the architecture's machine operation times for each state.
    \item \( cstoo \): The number of send operations that each client performs to the simulation executive for each request.
    \item \( artoo \): The number of send operations that the architecture performs to the simulation executive for each request.
    \item \( \tau \): The network transmission time, representing the time it takes to send a request over the network.
    \item \( \alpha \): A constant used for scaling time or representing additional delays in specific scenarios.
\end{itemize}

These variables form the basis for calculating the costs in both real-time and simulated-time simulations, allowing for a comprehensive analysis of the speed-up achieved using the simulation framework.

\section{Real-time Simulation Cost}

In real-time simulations, the system operates in synchrony with actual time, meaning that clients follow their defined behavior and sleep periods without any acceleration. The total real-time simulation cost, denoted as \( T_{real} \), can be decomposed into two main components: the time for client operations and the time for architecture operations.

\[
    T_{real} = TClient_{real} + TArchitecture_{real}
\]

The time for client operations \( TClient_{real} \) is given by:

\[
    TClient_{real} = \left\lceil \frac{p}{c} \right\rceil \cdot (svp \cdot cst) + \left\lceil \frac{p}{c} \right\rceil \cdot \tau + \left\lceil \frac{p}{c} \right\rceil \cdot (svp \cdot cmot)
\]

Where:
\begin{itemize}
    \item \( \left\lceil \frac{p}{c} \right\rceil \): The expected number of requests per client.
    \item \( svp \cdot cst \): The expected sleep times for a client, weighted by the stationary distribution vector \( svp \).
    \item \( \tau \): The network transmission time for a client to send a request to the server.
    \item \( svp \cdot cmot \): The expected time spent in machine operations by the client, weighted by the stationary distribution vector.
\end{itemize}

The time for architecture operations \( TArchitecture_{real} \) is given by:

\[
    TArchitecture_{real} = \left\lceil \frac{p}{c} \right\rceil \cdot (svp \cdot art) + \left\lceil \frac{p}{c} \right\rceil \cdot (svp \cdot ast) \cdot \tau + \left\lceil \frac{p}{c} \right\rceil \cdot (svp \cdot amot)
\]

Where:
\begin{itemize}
    \item \( svp \cdot art \): The expected server processing time for each request, weighted by the stationary distribution vector \( svp \).
    \item \( (svp \cdot ast) \cdot \tau \): The network time for sending responses to clients, where \( ast \) is the vector indicating the frequency of send operations based on the state.
    \item \( svp \cdot amot \): The expected time spent in machine operations by the architecture, weighted by the stationary distribution vector.
\end{itemize}

\section{Simulated-time Simulation Cost}

In simulated-time simulations, the system operates based on simulated time, skipping idle periods and thereby accelerating the simulation process. The total simulated-time simulation cost, denoted as \( T_{simulated} \), can be broken down into three main components: the time for client operations, the time for architecture operations, and the time for simulation executive operations.

\[
    T_{simulated} = TClient_{simulated} + TArchitecture_{simulated} + TSimExecutive_{simulated}
\]

The time for client operations \( TClient_{simulated} \) is given by:

\[
    TClient_{simulated} = (svp \cdot cstoo) \cdot p \cdot \tau + p \cdot \tau + p \cdot (svp \cdot cmot)
\]

Where:
\begin{itemize}
    \item \( (svp \cdot cstoo) \cdot p \cdot \tau \): The total network time for messages from clients to the simulation executive, considering the number of requests \( p \) and the network transmission time \( \tau \).
    \item \( p \cdot \tau \): The network time for sending requests to the server.
    \item \( p \cdot (svp \cdot cmot) \): The time spent in machine operations by the clients.
\end{itemize}

The time for architecture operations \( TArchitecture_{simulated} \) is given by:

\[
    TArchitecture_{simulated} = (svp \cdot artoo) \cdot p \cdot \tau + p \cdot (svp \cdot ast) \cdot \tau + p \cdot (svp \cdot amot)
\]

Where:
\begin{itemize}
    \item \( (svp \cdot artoo) \cdot p \cdot \tau \): The total network time for messages from the architecture to the simulation executive.
    \item \( p \cdot (svp \cdot ast) \cdot \tau \): The network time for sending responses to clients and internal messages within the architecture.
    \item \( p \cdot (svp \cdot amot) \): The time spent in machine operations by the architecture.
\end{itemize}

The time for simulation executive operations \( TSimExecutive_{simulated} \) is given by:

\[
    TSimExecutive_{simulated} = (svp \cdot (cstoo + artoo)) \cdot p \cdot \theta + p \cdot (\beta + \tau)
\]

Where:
\begin{itemize}
    \item \( (svp \cdot (cstoo + artoo)) \cdot p \cdot \theta \): The processing time of a request by the simulation executive, considering the combined send operations of clients and the architecture.
    \item \( p \cdot (\beta + \tau) \): The time to resolve synchronization and network delays in the system.
\end{itemize}

\section{Simulation Speed-up Factor}

The speed-up factor \( SimulationSpeedUp \) achieved by using simulated time can be estimated as:

\[
SimulationSpeedUp = \frac{T_{real}}{T_{simulated}}
\]

This ratio quantifies the extent to which the simulation runs faster by utilizing simulated time, eliminating idle periods, and optimizing request handling through the simulation executive.
