\begin{abstract}
This thesis focuses on the validation of a newly designed e-commerce system, with particular emphasis on non-functional requirements (NFRs) related to performance. The proposed architecture consists of multiple replications of the system, with a load balancer that distributes client requests among different servers. Client behavior is modeled using a Markov chain, simulating typical customer interactions with the system. The primary goal of this validation is to determine how well different architectural designs meet performance requirements under various loads.

To achieve this, a series of experiments were conducted to measure critical performance metrics, such as latency, response time, and the overall speed-up achieved through simulation. We evaluated the system under different configurations, varying the number of clients and the volume of requests. The results revealed that simulated-time simulations tend to be more efficient with a smaller number of clients, where the overhead of real-time coordination is minimized. Conversely, as the number of clients increases, real-time simulations become more effective, distributing the workload more evenly and improving performance.

Additionally, we investigated the system's capacity to handle 100 million requests while maintaining a speed-up of at least 10x. The findings suggest that approximately 2.9 million clients are required to achieve this level of performance. This research provides valuable insights into the trade-offs between simulated-time and real-time testing, offering practical guidance for system architects on how to optimize performance testing in large-scale e-commerce environments.
\end{abstract}